\documentclass[12]{tesis_tecnologica}

\usepackage[spanish]{babel}
\usepackage[utf8]{inputenc}
\usepackage[a4paper, total={15.5cm, 24.7cm}]{geometry}

\usepackage{appendix}
\usepackage{graphicx}
\graphicspath{ {Imagenes/} }

\usepackage{xcolor}

\title{[TÍTULO DE LA TESIS DE INVESTIGACIÓN TECNOLÓGICA]}
\author{WardoPo}
\assessor{Melkian}
\date{March 2021}

\makeglossaries
\newglossaryentry{abstract}
{
        name=abstarct,
        description={Voz inglesa que significa ‘breve resumen de un artículo científico, una ponencia o una comunicación, que suele publicarse junto con el texto completo’}
}
\newacronym{b}{B}{Flavia Alejandra Cervantes}

\begin{document}

\begin{titlepage}

    \includegraphics[width=0.4\textwidth]{La Salle Logo.png}
    
        \begin{center}
            \vspace*{1cm}
            \large{\textbf{ESCUELA PROFESIONAL DE INGENIERÍA DE SOFTWARE}}
            
            
            \vspace{1.5cm}
            \makeatletter
            \large{\@title}
            \makeatother
            
            
            \vspace{1.5cm}
            \makeatletter
            \@author
            \makeatother
            
            \vspace{1.5cm}
            \makeatletter
            \@assessor
            \makeatother
            
            \vspace{1.5cm}
            [Grado o Título a obtener]
            
            \vfill
            
            AREQUIPA – PERÚ
            
            \vspace{0.5cm}
            \makeatletter
            \@date
            \makeatother
        \end{center}

\end{titlepage}

\begin{dedicatoria}
A las Rosas, a la Luna.
Y a \acrlong{b}.
\end{dedicatoria}

\begin{agradecimientos}
A \acrshort{b}.
\end{agradecimientos}

\tableofcontents
\printglossary[type=\acronymtype]
\listoffigures
\listoftables
\printglossary

\begin{resumen}
El resumen posee un conjunto de elementos los cuales dan una visión clara del trabajo. El resumen debe contener lo siguiente: 

\begin{itemize}
    \item Delimitación de la Investigación.
    \item El propósito de la investigación. 
    \item Forma de lograrlo. 
    \item Criterios que justifican al estudio. 
    \item Fundamentación teórica empleada en el estudio. 
    \item Metodología de investigación empleada, se hace mención al tipo de estudio, el diseño de investigación, la modalidad de investigación (si es necesario),  técnica empleada para la recolección de datos, tipo de instrumento de recolección usado, validez y confiabilidad.
    \item Muy breve referencia a los resultados. 
    \item Señalamiento de las conclusiones más significativas 
    \item La forma de redacción es en pasado impersonal, por ejemplo: una vez que recabamos la información (esta forma es incorrecta), en vez de ello se debe decir: una vez que se recabó la información (estas la forma correcta). 
\end{itemize}

\end{resumen}

\begin{abstract}
    
    \Gls{abstract}
    
\end{abstract}

\begin{palabras_clave}

\end{palabras_clave}

\chapter{Problemática del Proyecto}

Es la identificación de la problemática que se trata de solucionar por medio de la investigación y, para la tesis, es en sí la elección del tema que servirá de base para elaborarla mediante una proposición concreta indicando el contexto preliminar del problema, definición del problema, antecedentes y/o estado del arte preliminar, objetivos, justificación/relevancia.

\section{Contexto del problema}

\section{Antecedentes y/o estado del arte}

En este apartado, considerado también como marco de referencia o estado del arte, se deberá analizar todo aquello que se ha escrito acerca del objeto de estudio: ¿qué se sabe del tema? ¿qué estudios se han hecho en relación a él? ¿desde qué perspectivas se ha abordado?\newline

Se debe evitar ahondar en teorías que sólo planteen un solo aspecto del fenómeno.\newline

Las funciones de los antecedentes son:

\begin{itemize}
    \item Delimitar el área de investigación
    \item Hacer un compendio de estudios realizados relacionados al tema de investigación
    \item Ayudar a prevenir errores que se han cometido en otros estudios
    \item Orientar sobre cómo habrá de llevarse a cabo el estudio
    \item Proveer un marco de referencia para interpretar los resultados del estudio
\end{itemize}

Esto implica realizar una revisión crítica de la literatura correspondiente, pertinente y actualizada. Al final, es importante fijar una determinada postura ante el fenómeno en cuestión.

\section{Definición del problema}

El problema deberá cumplir una serie de condiciones que de alguna forma justifiquen el esfuerzo necesario para resolverlo. Entre ellas: originalidad, trascendencia, actualidad, relevancia y la posibilidad de permitir el uso de lo aprendido a lo largo de la carrera. Se describe la situación problemática del contexto desde la perspectiva tecnológica. Debe enunciarse referencias que sustenten la problemática. Los problemas pueden formularse como preguntas o de manera declarativa.
\chapter{Planteamiento del Proyecto}

\section{Fundamentos teóricos}
Este apartado expone la sustentación teórica del problema de investigación u objeto de estudio, realizando un compendio de conocimientos existentes en el área que se va a investigar, expresando proposiciones teóricas.

\section{Objetivos del Proyecto}
\begin{subseccion}{Objetivo General}
Descripción de la finalidad principal que persigue la investigación, el motivo que le dará vigencia. 
\end{subseccion}

\begin{subseccion}{Objetivos Específicos}
Señalan las actividades que se deben cumplir para avanzar en la investigación y lo que se pretende lograr en cada una de las etapas de ella, por ende, la suma de los resultados de cada uno de los objetivos específicos permitirá alcanzar el propósito integral del objetivo general. Especifica los logros concatenados que se pretende conseguir.\newline

Para la formulación de los objetivos considere lo siguiente:

\begin{itemize}
    \item Deben estar dirigidos a los elementos básicos del problema
    \item Deben ser medibles y observables
    \item Deben ser claros y precisos
    \item Su formulación debe involucrar resultados concretos
    \item El alcance de los objetivos debe estar dentro de las posibilidades del investigador
    \item Deben ser expresados en verbos en infinitivos
\end{itemize}

\end{subseccion}

\section{Justificación}
Expone de manera lógica aspectos como:
\begin{itemize}
    \item Importancia de la investigación.
    \item Conveniencia del estudio.
    \item Aportes/beneficios al dominio.
    \item Implicación práctica.
    \item Utilidad metodológica.
\end{itemize} 

\section{Viabilidad}
Describir claramente las viabilidades económicas, técnicas y operativas que son necesarias para desarrollar la tesis

\section{Limitaciones}
Se describen las posibles implicancias y/o dificultades que limiten el correcto desarrollo de la tesis, afectando el alcance, viabilidad, objetivos, entre otros aspectos.

\chapter{Metodología de Desarrollo}

Se describen las etapas y actividades de acuerdo a la metodología de desarrollo escogida. Se enuncia el tipo de metodología y se explican de forma detallada los procedimientos, procesos o actividades de las que consta.\newline

Se debe de considerar la aplicación de técnicas de análisis de datos para la validación del trabajo.

\chapter{Resultados y Discusión}

Mostrar los resultados obtenidos como datos cualitativos o cuantitativos y realizar el análisis correspondiente de tal forma que se atienda el problema de investigación. Las evidencias que confirman las afirmaciones deben ser expuestas.\cite{einstein}

\begin{conclusiones}
Es la parte donde se manifiesta lo más destacado encontrado durante su investigación. Es una parte muy importante de la tesis puesto que en ella se indican los hallazgos y, en consecuencia, la comprobación de los objetivos. Aquí se muestran las aportaciones a la disciplina de estudio.\newline

Es recomendable que el alumno elabore sus conclusiones tratando de cubrir por lo menos algunos de los siguientes puntos, mismos que sólo se presentan como una guía o referencia para una mejor elaboración de esas conclusiones, pero no se describen en el texto. Estas sugerencias se enfocan hacia:

\begin{itemize}
    \item Resultados encontrados.
    \item Demostración realizada.
    \item Conclusión general.
    \item Conclusiones parciales (útiles al trabajo).
    \item Aportaciones a su tema (disciplina).
\end{itemize}

Algunas consideraciones a tener en cuenta son:

\begin{itemize}
    \item Evitar que las conclusiones sean un resumen de cada capítulo pues en realidad se trata de consecuencias y determinaciones que conllevan una verdad.
    \item Su redacción debe ser clara, concreta, directa y enfática.
    \item Es importante hacer conclusiones específicas por cada punto de interés, pero sin abusar de este recurso. Todo estará en función al tema, su importancia y lo relevante de lo encontrado.
    \item Llegar a una conclusión global, si es posible, en que se concentren los aspectos fundamentales de la investigación, procurando abarcar solamente lo sustantivo y básico del tema.
    \item Evitar el tono imperativo e impositivo tanto como el tímido y desobligado.
\end{itemize}


\end{conclusiones}

\bibliographystyle{IEEEtran}
\bibliography{bibliografia}

\appendix
\chapter*{Apendice 1: Ejemplo}
\label{annex:example}
Aqui va el contenido del apéndice

\end{document}
