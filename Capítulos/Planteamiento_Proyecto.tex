\chapter{Planteamiento del Proyecto}

\section{Fundamentos teóricos}
Este apartado expone la sustentación teórica del problema de investigación u objeto de estudio, realizando un compendio de conocimientos existentes en el área que se va a investigar, expresando proposiciones teóricas.

\section{Objetivos del Proyecto}
\begin{subseccion}{Objetivo General}
Descripción de la finalidad principal que persigue la investigación, el motivo que le dará vigencia. 
\end{subseccion}

\begin{subseccion}{Objetivos Específicos}
Señalan las actividades que se deben cumplir para avanzar en la investigación y lo que se pretende lograr en cada una de las etapas de ella, por ende, la suma de los resultados de cada uno de los objetivos específicos permitirá alcanzar el propósito integral del objetivo general. Especifica los logros concatenados que se pretende conseguir.\newline

Para la formulación de los objetivos considere lo siguiente:

\begin{itemize}
    \item Deben estar dirigidos a los elementos básicos del problema
    \item Deben ser medibles y observables
    \item Deben ser claros y precisos
    \item Su formulación debe involucrar resultados concretos
    \item El alcance de los objetivos debe estar dentro de las posibilidades del investigador
    \item Deben ser expresados en verbos en infinitivos
\end{itemize}

\end{subseccion}

\section{Justificación}
Expone de manera lógica aspectos como:
\begin{itemize}
    \item Importancia de la investigación.
    \item Conveniencia del estudio.
    \item Aportes/beneficios al dominio.
    \item Implicación práctica.
    \item Utilidad metodológica.
\end{itemize} 

\section{Viabilidad}
Describir claramente las viabilidades económicas, técnicas y operativas que son necesarias para desarrollar la tesis

\section{Limitaciones}
Se describen las posibles implicancias y/o dificultades que limiten el correcto desarrollo de la tesis, afectando el alcance, viabilidad, objetivos, entre otros aspectos.
