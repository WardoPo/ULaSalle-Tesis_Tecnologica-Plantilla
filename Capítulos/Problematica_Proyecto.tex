\chapter{Problemática del Proyecto}

Es la identificación de la problemática que se trata de solucionar por medio de la investigación y, para la tesis, es en sí la elección del tema que servirá de base para elaborarla mediante una proposición concreta indicando el contexto preliminar del problema, definición del problema, antecedentes y/o estado del arte preliminar, objetivos, justificación/relevancia.

\section{Contexto del problema}

\section{Antecedentes y/o estado del arte}

En este apartado, considerado también como marco de referencia o estado del arte, se deberá analizar todo aquello que se ha escrito acerca del objeto de estudio: ¿qué se sabe del tema? ¿qué estudios se han hecho en relación a él? ¿desde qué perspectivas se ha abordado?\newline

Se debe evitar ahondar en teorías que sólo planteen un solo aspecto del fenómeno.\newline

Las funciones de los antecedentes son:

\begin{itemize}
    \item Delimitar el área de investigación
    \item Hacer un compendio de estudios realizados relacionados al tema de investigación
    \item Ayudar a prevenir errores que se han cometido en otros estudios
    \item Orientar sobre cómo habrá de llevarse a cabo el estudio
    \item Proveer un marco de referencia para interpretar los resultados del estudio
\end{itemize}

Esto implica realizar una revisión crítica de la literatura correspondiente, pertinente y actualizada. Al final, es importante fijar una determinada postura ante el fenómeno en cuestión.

\section{Definición del problema}

El problema deberá cumplir una serie de condiciones que de alguna forma justifiquen el esfuerzo necesario para resolverlo. Entre ellas: originalidad, trascendencia, actualidad, relevancia y la posibilidad de permitir el uso de lo aprendido a lo largo de la carrera. Se describe la situación problemática del contexto desde la perspectiva tecnológica. Debe enunciarse referencias que sustenten la problemática. Los problemas pueden formularse como preguntas o de manera declarativa.